%
% Complete documentation on the extended LaTeX markup used for Insight
% documentation is available in ``Documenting Insight'', which is part
% of the standard documentation for Insight.  It may be found online
% at:
%
%     http://www.itk.org/

\documentclass{InsightArticle}


%%%%%%%%%%%%%%%%%%%%%%%%%%%%%%%%%%%%%%%%%%%%%%%%%%%%%%%%%%%%%%%%%%
%
%  hyperref should be the last package to be loaded.
%
%%%%%%%%%%%%%%%%%%%%%%%%%%%%%%%%%%%%%%%%%%%%%%%%%%%%%%%%%%%%%%%%%%
\usepackage[dvips,
bookmarks,
bookmarksopen,
backref,
colorlinks,linkcolor={blue},citecolor={blue},urlcolor={blue},
]{hyperref}
% to be able to use options in graphics
\usepackage{graphicx}
% for pseudo code
\usepackage{listings}
% subfigures
\usepackage{subfigure}


%  This is a template for Papers to the Insight Journal. 
%  It is comparable to a technical report format.

% The title should be descriptive enough for people to be able to find
% the relevant document. 
\title{Easily resample anisotropic image to make it isotropic}

% Increment the release number whenever significant changes are made.
% The author and/or editor can define 'significant' however they like.
% \release{0.00}

% At minimum, give your name and an email address.  You can include a
% snail-mail address if you like.
\author{Ga\"etan Lehmann}
\authoraddress{Unit\'e de Biologie du D\'eveloppement et de la Reproduction, Institut National de la Recherche Agronomique, 78350 Jouy-en-Josas, France}

\begin{document}
\maketitle

\ifhtml
\chapter*{Front Matter\label{front}}
\fi


\begin{abstract}
\noindent
Anisotropic images are quite common, and are often difficult to manipulate. The filter proposed here make it simple to transform an anisotropic image in an isotropic one.
\end{abstract}

% \tableofcontents

\section{Introduction}

Even if the 3D imaging is more and more common, lots of images still have anisotropic resolution. With confocal microscopy for example, the resolution is still lower on the axe z than on the axes x and y.

The filter proposed here let the user easily produce a isotropic image from an anisotropic one, while some options let him choose which method to use, and limit the size of the output image.

\section{Implementation}

IsotropicResolutionImageFilter is built as a sequence of filters. Everything done is fully feasible without this filter - IsotropicResolutionImageFilter is just there to make the configuration of ResampleImageFilter easier, and to make the integration in a pipeline easier.

The by default, the filter will use linear interpolation to compute the new pixels, and will smooth the image with a gaussian filter if the number of pixel is decrease on an axe to avoid aliasing effects. This is well described in the ITK Software Guide, chapter 6.9~\cite{ITKSoftwareGuide}~\footnote{However, contrary to what is said in the ITK Software Guide, you shouldn't feel so guilty to resample your anisotropic images. Yes, sometime, it's just the best thing to do.}.

The user can also choose to use only the nearest neighbors. This method should be used with synthetics images, to avoid introducing new labels for example.

\section{Sample code}
Here is the code of a simple program which read an image, transform it to an isotropic image and save it.

\small \begin{verbatim}
#include "itkImageFileReader.h"
#include "itkImageFileWriter.h"
#include "itkSimpleFilterWatcher.h"

#include "itkIsotropicResolutionImageFilter.h"


int main(int, char * argv[])
{
  const int dim = 3;
  
  typedef unsigned char PType;
  typedef itk::Image< PType, dim > IType;

  typedef itk::ImageFileReader< IType > ReaderType;
  ReaderType::Pointer reader = ReaderType::New();
  reader->SetFileName( argv[3] );

  typedef itk::IsotropicResolutionImageFilter< IType, IType > FilterType;
  FilterType::Pointer filter = FilterType::New();
  filter->SetInput( reader->GetOutput() );
  filter->SetMaximumIncrease( atof(argv[1]) );
  filter->SetNearestNeighbor( atoi(argv[2]) );

  itk::SimpleFilterWatcher watcher(filter, "filter");

  typedef itk::ImageFileWriter< IType > WriterType;
  WriterType::Pointer writer = WriterType::New();
  writer->SetInput( filter->GetOutput() );
  writer->SetFileName( argv[4] );
  writer->Update();

  return 0;
}
\end{verbatim} \normalsize


\section{Acknowledgments}
We thank Dr Pierre Adenot and MIMA2 confocal facilities
(\url{http://mima2.jouy.inra.fr})
for providing image samples.


\appendix



\bibliographystyle{plain}
\bibliography{InsightJournal}
\nocite{ITKSoftwareGuide}

\end{document}

